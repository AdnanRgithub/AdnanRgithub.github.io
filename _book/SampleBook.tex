\documentclass[]{book}
\usepackage{lmodern}
\usepackage{amssymb,amsmath}
\usepackage{ifxetex,ifluatex}
\usepackage{fixltx2e} % provides \textsubscript
\ifnum 0\ifxetex 1\fi\ifluatex 1\fi=0 % if pdftex
  \usepackage[T1]{fontenc}
  \usepackage[utf8]{inputenc}
\else % if luatex or xelatex
  \ifxetex
    \usepackage{mathspec}
  \else
    \usepackage{fontspec}
  \fi
  \defaultfontfeatures{Ligatures=TeX,Scale=MatchLowercase}
\fi
% use upquote if available, for straight quotes in verbatim environments
\IfFileExists{upquote.sty}{\usepackage{upquote}}{}
% use microtype if available
\IfFileExists{microtype.sty}{%
\usepackage{microtype}
\UseMicrotypeSet[protrusion]{basicmath} % disable protrusion for tt fonts
}{}
\usepackage[margin=1in]{geometry}
\usepackage{hyperref}
\hypersetup{unicode=true,
            pdftitle={Sample Blog},
            pdfauthor={Author Name},
            pdfborder={0 0 0},
            breaklinks=true}
\urlstyle{same}  % don't use monospace font for urls
\usepackage{natbib}
\bibliographystyle{apalike}
\usepackage{longtable,booktabs}
\usepackage{graphicx,grffile}
\makeatletter
\def\maxwidth{\ifdim\Gin@nat@width>\linewidth\linewidth\else\Gin@nat@width\fi}
\def\maxheight{\ifdim\Gin@nat@height>\textheight\textheight\else\Gin@nat@height\fi}
\makeatother
% Scale images if necessary, so that they will not overflow the page
% margins by default, and it is still possible to overwrite the defaults
% using explicit options in \includegraphics[width, height, ...]{}
\setkeys{Gin}{width=\maxwidth,height=\maxheight,keepaspectratio}
\IfFileExists{parskip.sty}{%
\usepackage{parskip}
}{% else
\setlength{\parindent}{0pt}
\setlength{\parskip}{6pt plus 2pt minus 1pt}
}
\setlength{\emergencystretch}{3em}  % prevent overfull lines
\providecommand{\tightlist}{%
  \setlength{\itemsep}{0pt}\setlength{\parskip}{0pt}}
\setcounter{secnumdepth}{5}
% Redefines (sub)paragraphs to behave more like sections
\ifx\paragraph\undefined\else
\let\oldparagraph\paragraph
\renewcommand{\paragraph}[1]{\oldparagraph{#1}\mbox{}}
\fi
\ifx\subparagraph\undefined\else
\let\oldsubparagraph\subparagraph
\renewcommand{\subparagraph}[1]{\oldsubparagraph{#1}\mbox{}}
\fi

%%% Use protect on footnotes to avoid problems with footnotes in titles
\let\rmarkdownfootnote\footnote%
\def\footnote{\protect\rmarkdownfootnote}

%%% Change title format to be more compact
\usepackage{titling}

% Create subtitle command for use in maketitle
\newcommand{\subtitle}[1]{
  \posttitle{
    \begin{center}\large#1\end{center}
    }
}

\setlength{\droptitle}{-2em}

  \title{Sample Blog}
    \pretitle{\vspace{\droptitle}\centering\huge}
  \posttitle{\par}
    \author{Author Name}
    \preauthor{\centering\large\emph}
  \postauthor{\par}
      \predate{\centering\large\emph}
  \postdate{\par}
    \date{2019-01-19}

\usepackage{booktabs}

\begin{document}
\maketitle

{
\setcounter{tocdepth}{1}
\tableofcontents
}
\chapter{Introduction}\label{introduction}

Brief motivation behind the blog

\chapter{The magic of the central limit
theorem:}\label{the-magic-of-the-central-limit-theorem}

\section{Sampling, sampling,
sampling\ldots{}}\label{sampling-sampling-sampling}

As scientists we aim to understand the world around us, not just our
immediate environments. In most cases, we don't have access to
\textbf{populations}, for one, because they are\ldots{} large. For
example, if you're studying expectant mothers, it is virtually
impossible to collect data from all of the expectant mothers from around
the world. Therefore, we make do with random and representative
\emph{samples} of the population to make generalizations -- that is,
statements -- about the population as a whole.

The \textbf{central limit theorem (CLT)} states that the larger the
sample size collected, the closer the distribution of the \emph{sample}
means will resemble a normal distribution (\emph{bell curve}),
regardless of the population's distribution. \emph{If you were asleep
during your stats class or need a refresher, Khan Academy gives a good
introduction to CLT.}

It's useful to re-read the statement above, because we're not saying
that we're assuming that the observations (e.g., one sample of with
\(100\)) originate from a normal distribution. We're saying that the
\emph{distribution of sample means} (based on taking the mean of many
separate samples that originate from the ``parent'' distribution) will
be normally distributed. That is, provided your sample size is ``large
enough.'' The theorem works ``in the limit'', as mathematicians say, but
a general rule is that the sample size should be at least \(30\) for the
CLT to hold for almost any data distribution. And, in practice, it will
work on smaller samples if they originate from a population that
actually has a normal distribution.

Let's refresh ourselves with a few terms. \textbf{Variance} refers to
the amount of variability, or how spread the data are from the mean. The
population variance is symbolized as \textbf{sigma squared}, or
\(\sigma^{2}\). Because variance is a squared term, we tend to look at
the square root of variance, or \textbf{standard deviation}, and the
population standard deviation is symbolized as \(\sigma\), or
\textbf{sigma}.

We know that as the size of the sample increases, the closer the
\emph{sampling distribution of the sample mean will resemble a bell
curve with a mean} that approaches the population mean, \(\mu\). How
about the standard deviation of the distribution of the sample means? As
the sample size increases, the CLT says that the standard deviation will
approach \(\frac{\mathbf{\sigma}}{\sqrt{n}}\). Again, this results holds
despite the shape of the population distribution.

A good source of intuitive discussion on the central limit theorem is
\href{http://www2.psychology.uiowa.edu/faculty/mordkoff/GradStats/part\%201/I.07\%20normal.pdf}{Mordkoff,
J.T. (2016) The Assumption(s) of Normality}.

\begin{center}\rule{0.5\linewidth}{\linethickness}\end{center}

Post originally by Kelly Morrow and edited by L. Pessoa

\chapter{Applying the Central Limit Theorem}\label{CLT_applying}

\section{An example}\label{an-example}

According the National Center for Health Statistics, the distribution of
serum cholesterol levels for 20 to 74-year-old males living in the
United States has mean 211 mg/dl, and a standard deviation of 46 mg/dl.
We are planning to collect a sample of 25 individuals and measure their
cholesterol levels.

We are interested in the following about the sample:

\begin{enumerate}
\def\labelenumi{\arabic{enumi}.}
\tightlist
\item
  What is the probability that our sample mean will be above a certain
  limit, say 230?
\item
  What is the 95\% confidence interval of our sample means?
\item
  How do these vary as we collect more samples? Does the probability
  increase or decrease? Does the size of the confidence interval
  increase or decrease? By what factor does it increase or decrease?
\item
  Finally, how large would the sample size have to be to ensure a 95\%
  probability that the sample average is within 5 mg/dl of the
  population mean?
\end{enumerate}

\emph{How does the Central Limit Theorem (CLT) help us answer these
questions?}

\begin{center}\rule{0.5\linewidth}{\linethickness}\end{center}

\section{Central Limit Theorem}\label{central-limit-theorem}

Given a population with a finite mean \(\mu\) and a finite non-zero
standard deviation \(\sigma\), the distribution of the sample means
approaches a normal distribution as the sample size increases.

The mean of the sample means is given by

\[\mu_{\bar{X}} = \mu,\]

and the standard deviation of the sample means (also referred to as the
standard error of the mean) is given by

\[\sigma_{\bar{X}} = \frac{\sigma}{\sqrt{N}}.\]

For a more precise version of CLT, please refer
\href{http://mathworld.wolfram.com/CentralLimitTheorem.html}{Wolfram}.

An important observation is that no assumptions are made about the
distribution of the parental population. It could be discrete or
continuous, severely skewed, but as long as the mean and standard
deviation are finite, CLT holds. To convince yourself of this, please
take a look at examples
\href{http://sphweb.bumc.bu.edu/otlt/MPH-Modules/BS/BS704_Probability/BS704_Probability12.html}{here}
or use the simulator
\href{http://onlinestatbook.com/stat_sim/sampling_dist/index.html}{here}.

\begin{center}\rule{0.5\linewidth}{\linethickness}\end{center}

\section{Samples}\label{samples}

But what is a \textbf{sample}? We keep using that word, and its meaning
is quite an important concept. A sample is a random draw of size N of
data from the parent distribution. We obtain a sample of data every time
we randomly draw from what we conceptualize as the population of
interest. If the population of interest in every student at UMD, then a
random draw is obtained by any mechanism that (truly) randomly draws
from it (think of an imaginary lottery machine that, after spinning it,
we can obtain one student at a time).

And the \textbf{sample mean}? The sample mean is just the average of the
measure of interest from the \(N\) units that were sampled. So for each
mean, we get exactly one sample mean.

When we're thinking about the CLT (what it means), we need to think of
repeating this process many times to have \emph{multiple sample means}.
Remember, that each sample has \(N\) units. So for each mean, you need
to sample a group of size \(N\). This is what the CLT talks about.

One reason this might appear confusing is that in any one given study,
we only sample once (with size \(N\)). That's the world the experimenter
lives in (except it she repeats her experiment). But that's not the
world of the CLT, which instead is a world in which we perform an
experiment (a sample draw of size \(N\)), over and over. And a few more
times\ldots{}

\begin{center}\rule{0.5\linewidth}{\linethickness}\end{center}

Back to the previous questions. To answer them, it is essential to know
the \emph{sampling distribution}, that is, the \emph{distribution} of
the sample mean.

\begin{itemize}
\tightlist
\item
  Since the standard deviation of the parent population is known, from
  CLT it follows that the sampling distribution (\(N=25\)) has a mean
  \(\mu_{\bar{X}} = 211\) mg/dl, and standard deviation (this is called
  standard error) \(\sigma_{\bar{X}} = \frac{46}{\sqrt{25}} = 9.2.\)
  Note that the standard error reduces as the number of samples increase
  by a factor \(\sqrt{N}.\)
\item
  Our limit, \(230\) mg/dl, is therefore
  \(\frac{230 - 211}{9.2} = 2.07\) standard deviations away from the
  mean. In other words, the z-score associated with the limit \(230\)
  mg/dl is \(2.07\).
\end{itemize}

\begin{enumerate}
\def\labelenumi{\arabic{enumi}.}
\tightlist
\item
  The answer to our first question is simple the probability that a
  normally distributed random variable is greater than \(2.07\) standard
  deviations away from the mean = \(1.9\%\) (or \(0.019\)).
\item
  For a normally distributed random variable, 95\% of the values lie
  within 1.96 standard deviations of its mean (on either side). The
  standard deviation remains the same, \(9.2\). Thus, the \(95\%\)
  confidence interval is simply, 211 - \(1.96(9.2) = 193.0\) to
  \(211 + 1.96(9.2) = 229.0\).
\item
  Suppose we had only \(10\) samples. Verify that the new standard error
  would be \(14.5\) and the z-score associated with the limit, \(230\)
  mg/dl, would be \(1.31\). The probability of our sample mean being
  over \(230\) would thus be \(9.6\%\) (almost \(5\) times higher). On
  the other hand, our confidence interval would be much larger with
  \(10\) samples; \((182.5, 239.5)\). How about if we had \(50\)
  samples? This would result is a narrower confidence interval
  \((198.2, 223.8)\) since the standard error is smaller.
\item
  To answer our final question, we need \(1.96\) standard deviations of
  the sampling distribution to amount to \(5\) mg/dl. Thus, the standard
  error should be
  \(\sigma_{\bar{X}} = \frac{\sigma}{\sqrt{N}} = 5/1.96.\) Since, we
  know \(\sigma\), \(N = 325.1\). We would need at least \(326\) samples
  to ensure this confidence interval.
\end{enumerate}

\begin{center}\rule{0.5\linewidth}{\linethickness}\end{center}

\section{Hypothesis Testing}\label{hypothesis-testing}

Thinking along these lines can be used to develop hypothesis tests and
understand p-values. We start again with an example:

Cystic fibrosis is a genetic disease that affects lung function. Forced
vital capacity (FVC) is the volume of air that a person can expel from
the lungs in \(6\) seconds. It is often used as a marker of the
progression of cystic fibrosis. \(14\) participants received both a drug
and placebo (at different times), and their FVC was measured at the
beginning and end of each treatment period. In the study, the mean
difference in reduction in FVC (placebo - drug) was \(137\), with a
sample standard deviation \(223\). Does the drug have a significant
effect?

The null hypothesis is that the drug has no effect, thus the reduction
in FVC should be zero. Let's first find the probability of observing an
FVC reduction of greater than \(137\) given the null hypothesis.

We calculate the standard error based on the \(14\) participants as
\(\sigma_{\bar{X}} = \frac{223}{\sqrt{14}} = 60.\) The z-score
associated with the mean reduction in FVC is given by
\(\frac{(137 - 0)}{60} = 2.28\). The probability of observing a value
further from the mean by at least \(2.28\) standard deviations, which is
also the p-value, which is \(2.2\%\) (or \(0.022\)).

This is a small probability that the drug is having an effect just by
chance. Why did we obtain a small probability? Because it's effect by
chance should be zero. But because we're working with a sample
(\(N=14\)) that is randomly drawn from the population, the observed mean
reduction will fluctuate from sample to sample. Based on data from our
sample, the reduction was \(137\). But how large is \(137\)? We don't
know without some form of calibration, which can be obtained by the
information that was given to us: \(\sigma_{\bar{X}}\).

From these data, we can believe that the drug helps prevent
deterioration in lung function. At least the data are consistent with
this notion in a probabilistic sense. Another way to think about this,
it would be somewhat irrational to think that we could obtain the
observed result just by chance. Maybe not entirely with a p of basically
\(\frac{2}{100}\) but probably with a p of \(\frac{1}{1000}\). But
obviously this is somewhat subjective.

\begin{center}\rule{0.5\linewidth}{\linethickness}\end{center}

\section{The flaw in the z-test}\label{the-flaw-in-the-z-test}

Is the above reasoning correct? To understand this we need to understand
the difference between the first and second examples.

In our first example, an oracle provided us with the standard deviation
of the population. Some all-knowing being told us what the population
\(\sigma\) was. But in the second example the standard error was based
on the sample. To make the distinction clear, we call the standard error
based on sample data \(s_{\bar{X}}\).

\emph{Important aside}: why use the sample and not the population?
Populations are essentially Platonic objects. We typically don't have
complete knowledge about the objects we want to study. If we did we
woulnd't need to study them in the first place! So we have to draw
samples and do the best based on finite amounts of data. Another way to
think about this is that oracles don't walk around waiting to be
interrogated. Maybe they were around in ancient Greece, but not
anymore\ldots{}

Gossett (which published under the pseudonym Student) showed that when
the standard error is estimated from sample data, the statistic
\(\frac{\bar{x} - \mu}{\sigma_{\bar{X}}}\) is not normal, but follows a
t distribution with \(N - 1\) degrees of freedom (df). The t
distribution looks very much like a normal, but has what we call heavier
tails, that is, more mass along the tails relative to the normal.

Thus, the probability associated with a t-score of \(2.28\) with
\(14 - 1\) degrees of freedom can be calculated to be \(4\%\) (or
\(0.04\)). The p-value obtained from the z-test (\(2.2\%\)) overstates
the evidence against the null hypothesis (this is consistent with the
fact that a normal is thinner along the tails than the t distribution).

\begin{center}\rule{0.5\linewidth}{\linethickness}\end{center}

Examples are borrowed from
\href{http://myweb.uiowa.edu/pbreheny/4120/s18/index.html}{Introduction
to Biostatistics} kindly offered by Patrick Breheny at UIowa.

Post by Manasij Venkatesh, with edits by L. Pessoa.

\bibliography{book.bib,packages.bib}


\end{document}
